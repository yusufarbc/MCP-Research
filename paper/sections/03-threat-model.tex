Varlıklar
- Gizli veriler: API anahtarları, kimlik bilgileri, müşteri/kurumsal veriler.
- Eylem bütünlüğü: araç çağrıları, dosya değişiklikleri, yürütme çıktıları.
- Kullanılabilirlik: istemci/sunucu çalışma süresi, oran ve bütçe sınırları.

Saldırganlar
- Ağ üzerinden dış aktör.
- Kötü niyetli/ele geçirilmiş araç veya sunucu sağlayıcı (tedarik zinciri).
- İç tehdit veya ele geçirilmiş işletim/uç nokta.

Varsayımlar
- Taşıma katmanı gizlilik/bütünlük sağlar (TLS), ancak uçlar güvenilir olmayabilir.
- Modeller halüsinasyon üretebilir; içerik veya araç meta verisi yoluyla prompt enjeksiyonu mümkündür.
- Araçların etki alanı farklıdır (dosya sistemi > salt-okunur HTTP gibi).

Güvenlik hedefleri
- Gizlilik: sırların araç yanıtları ve günlükler ile sızmaması.
- Bütünlük: yetkisiz/istenmeyen yan etkilerin önlenmesi.
- Kullanılabilirlik: kaynak kullanımını sınırla; saldırı altında güvenli bozulma.

Güven Sınırları
- İstemci $\leftrightarrow$ Sunucu: kimlik doğrulama, yetkilendirme, yetenek/kapasite mutabakatı.
- Sunucu $\leftrightarrow$ Araçlar: yalıtım ve politika uygulama.
- Kullanıcı $\leftrightarrow$ Model: prompt hijyeni, içerik filtreleme, kullanıcı onayı.

