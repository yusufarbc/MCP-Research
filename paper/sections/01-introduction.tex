MCP, Büyük Dil Modeli (LLM) uygulamalarının harici araçlar ve veri kaynaklarıyla etkileşimini standartlaştırmayı amaçlar. Bu, önemli yetenek ve birlikte çalışabilirlik faydaları sağlarken, geleneksel web API’leri ve eklenti ekosistemlerinden farklı güven sınırları ve saldırı yüzeyleri de getirir. Bu çalışma, MCP’nin güvenlik risklerine bütüncül bir bakış ve pratik sıkılaştırma rehberi sunar.

Katkılarımız
- Varlıklar, saldırgan tipleri, varsayımlar ve güvenlik hedeflerini netleştiren yapılandırılmış bir MCP tehdit modeli.
- Tool/prompt poisoning, plan enjeksiyonu, RCE ve SSRF/DNS rebinding ile tedarik zinciri risklerini içeren bir saldırı taksonomisi; güncel örneklerle birlikte \cite{arXiv250323278,arXiv250907595,arXiv250812566,MDPIElectronics3267}.
- Yalıtım ve sandboxing, en az ayrıcalık, sıkı girdi/çıktı doğrulaması, gözlemlenebilirlik ve oran sınırlama gibi uygulanabilir savunmalar ile protokol düzeyi öneriler.

Kapsam ve Kısıtlar
- Odak, modelin içsel eğitimi değil protokol kullanımı ve tipik yığınlardır (istemci–sunucu–taşıma).
- Örnekler, yanlış yapılandırma riskleri ve katmanlı savunmaya vurgu ile işletimsel (DevOps/SecOps) ihtiyaçları hedefler.

