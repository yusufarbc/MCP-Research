MCP, LLM tabanlı uygulamalar için araç ve veri erişimini ortak bir protokole soyutlayarak; MCP istemcisi (LLM uygulaması), bir veya daha fazla MCP sunucusu (araç/ kaynak vekili) ve harici araç/hizmet rollerini tanımlar. Taşıma çoğunlukla JSON-RPC benzeri kanallar üzerinden gerçekleşir.

Temel Bileşenler
- İstemci: İstek üretir, oturum bağlamını yönetir, UI/UX ilkelerini uygular.
- Sunucu: Araç ve veri kaynaklarını tekdüze bir arabirim üzerinden sunar, yürütmeyi aracılar.
- Araç/Hizmet: Dosya sistemi, HTTP API, veritabanı gibi farklı risk profillerine sahip eylemler.

Güven sınırları: İstemciler sunucu/araçları varsayılan olarak güvenilir saymamalı; sunucu araçları politikalarla kısıtlamalı; araçlar mümkün olduğunca yalıtılmalıdır.

\begin{figure}[h]
\centering
\begin{tikzpicture}[
  comp/.style={rectangle,draw,rounded corners,align=center,minimum width=3.2cm,minimum height=1cm},
  arr/.style={-Latex}
]
\node[comp, fill=blue!5] (client) {MCP İstemci\\(LLM Uyg)};
\node[comp, fill=green!5, right=3.2cm of client] (server) {MCP Sunucu};
\node[comp, fill=orange!10, above right=0.8cm and 3.2cm of server] (tool1) {Araç A};
\node[comp, fill=orange!10, below right=0.8cm and 3.2cm of server] (tool2) {Araç B};
\draw[arr] (client) -- node[above]{Taşıma} (server);
\draw[arr] (server) -- (tool1);
\draw[arr] (server) -- (tool2);
\end{tikzpicture}
\caption{Basitleştirilmiş MCP topolojisi ve güven sınırları.}
\label{fig:mcp-arch}
\end{figure}

Operasyonel Bağlam
- Kurulumlar yerel geliştirici ortamlarından (yüksek ayrıcalık, düşük yalıtım) çok kiracılı bulut dağıtımlarına (sıkı yalıtım ve denetleme) kadar çeşitlenir.
- Açık sunucu ve zayıf kimlik doğrulama/yetkilendirme gibi yanlış yapılandırmalar maruziyeti ciddi biçimde artırır \cite{MDPIElectronics3267}.

