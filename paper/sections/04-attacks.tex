We group attacks by their primary vector and intended impact, illustrating each with recent cases.

Prompt/Tool Poisoning
- Malicious instructions embedded in tool descriptions or responses steer the model to leak data or execute risky actions. Impacts include credential exfiltration and policy bypass \cite{arXiv250323278,arXiv250812566}.

Plan Injection
- Multi-step reasoning frameworks (ReAct, ToT) can be subverted by crafted intermediate content to alter subsequent actions, escalating privileges over iterations \cite{arXiv250907595}.

Remote Code Execution (RCE)
- Unsafe command construction, insecure deserialization, or over-privileged tools lead to arbitrary code execution on the server/host.

Network Pivoting (SSRF/DNS Rebinding)
- Tools performing network I/O may be abused to access internal services or metadata endpoints; DNS rebinding can break origin assumptions.

AuthN/Z Gaps and Open Servers
- Exposed MCP servers without proper authentication or capability scoping allow arbitrary queries and data access \cite{MDPIElectronics3267}.

Supply-Chain and Typosquatting
- Malicious packages or server updates inject backdoors into tools, compromising downstream clients.

\begin{table}[h]
\centering
\begin{tabular}{@{}lll@{}}
\toprule
Attack Class & Typical Vector & Primary Impact \\
\midrule
Tool/Prompt Poisoning & Untrusted metadata/content & Data exfiltration, policy bypass \\
Plan Injection & Iterative reasoning artifacts & Unauthorized actions, escalation \\
RCE & Over-privileged tools, unsafe exec & Host compromise \\
SSRF/DNS Rebinding & Network-capable tools & Internal data access, pivot \\
AuthN/Z Gaps & Open/weakly protected servers & Unrestricted queries \\
Supply Chain & Malicious packages/updates & Widespread compromise \\
\bottomrule
\end{tabular}
\caption{MCP attack classes and effects (non-exhaustive).}
\label{tab:attacks}
\end{table}

