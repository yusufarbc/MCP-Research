Saldırı sınıflarını ana vektör ve etkilerine göre grupluyor, güncel vakalarla örneklendiriyoruz.

Prompt/Tool Poisoning
- Araç tanımlarına veya yanıtlarına gömülü kötü talimatlar, modeli veri sızıntısı veya riskli eylemlere yöneltebilir. Etkiler: kimlik bilgisi sızdırma, politika atlatma \cite{arXiv250323278,arXiv250812566}.

Plan Enjeksiyonu
- ReAct, Tree-of-Thoughts gibi çok adımlı akıl yürütme yöntemleri; ara içeriklerle sonraki adımların saptırılması ve kademeli ayrıcalık artışı için hedef alınabilir \cite{arXiv250907595}.

Uzaktan Kod Çalıştırma (RCE)
- Güvensiz komut inşası, serileştirme veya aşırı ayrıcalıklı araçlar, sunucu/host üzerinde keyfi kod çalıştırmaya yol açar.

Ağ Pivotu (SSRF/DNS Rebinding)
- Ağ I/O yapan araçlar, iç servisler veya metadata uç noktalarına erişim için kötüye kullanılabilir; DNS rebinding köken varsayımlarını bozar.

Kimlik Doğrulama/Yetkilendirme Açıkları ve Açık Sunucular
- Kimlik doğrulaması veya yetenek kapsamı olmadan erişime açılan MCP sunucuları ile sınırsız sorgu ve veri erişimi mümkün olur \cite{MDPIElectronics3267}.

Tedarik Zinciri ve Typosquatting
- Kötü amaçlı paketler veya sunucu güncellemeleri arka kapı yerleştirebilir; aşağı akıştaki istemcileri etkiler.

\begin{table}[h]
\centering
\begin{tabular}{@{}lll@{}}
\toprule
Saldırı Sınıfı & Tipik Vektör & Birincil Etki \\
\midrule
Tool/Prompt Poisoning & Güvenilmeyen meta veri/içerik & Veri sızıntısı, politika atlatma \\
Plan Enjeksiyonu & Yinelemeli akıl yürütme çıktıları & Yetkisiz eylem, ayrıcalık artışı \\
RCE & Aşırı ayrıcalıklı araçlar, güvensiz exec & Host ele geçirilmesi \\
SSRF/DNS Rebinding & Ağ yetenekli araçlar & İç veri erişimi, pivot \\
AuthN/Z Açıkları & Açık/zayıf korumalı sunucular & Sınırsız sorgu \\
Tedarik Zinciri & Kötü paketler/güncellemeler & Yaygın kompromizasyon \\
\bottomrule
\end{tabular}
\caption{MCP saldırı sınıfları ve etkileri (tam değildir).}
\label{tab:attacks}
\end{table}

